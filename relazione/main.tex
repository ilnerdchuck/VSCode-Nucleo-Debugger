\documentclass[a4paper, twoside,openright]{report}

\usepackage[a4paper,top=3cm,bottom=3cm,left=3cm,right=3cm]{geometry}
\usepackage[fontsize=13pt]{scrextend}
\usepackage[english,italian]{babel}
\usepackage[fixlanguage]{babelbib}
\usepackage[utf8]{inputenc}
\usepackage[T1]{fontenc}
\usepackage{lipsum}
\usepackage{rotating}
\usepackage{fancyhdr}
\usepackage{amssymb}
\usepackage{amsmath}
\usepackage{amsthm}
\usepackage{lmodern}
\usepackage{graphicx}
\usepackage[dvipsnames]{xcolor}
\usepackage{listings}
\usepackage{listings}
\usepackage{xparse}
\usepackage{hyperref}
\usepackage[normalem]{ulem}
\usepackage{float}


\definecolor{dkgreen}{rgb}{0,0.6,0}
\definecolor{gray}{rgb}{0.5,0.5,0.5}
\definecolor{mauve}{rgb}{0.58,0,0.82}

\lstset{frame=tb,
  language=Java,
  aboveskip=3mm,
  belowskip=3mm,
  showstringspaces=false,
  columns=flexible,
  basicstyle={\small\ttfamily},
  numbers=none,
  numberstyle=\tiny\color{gray},
  keywordstyle=\color{blue},
  commentstyle=\color{dkgreen},
  stringstyle=\color{mauve},
  breaklines=true,
  breakatwhitespace=true,
  tabsize=3
}

\lstdefinelanguage{JavaScript}{
  keywords={typeof, new, true, false, catch, function, return, null, catch, switch, var, if, in, while, do, else, case, break},
  keywordstyle=\color{blue}\bfseries,
  ndkeywords={class, export, boolean, throw, implements, import, this},
  ndkeywordstyle=\color{darkgray}\bfseries,
  identifierstyle=\color{black},
  sensitive=false,
  comment=[l]{//},
  morecomment=[s]{/*}{*/},
  commentstyle=\color{purple}\ttfamily,
  stringstyle=\color{red}\ttfamily,
  morestring=[b]',
  morestring=[b]"
}

\lstset{
   language=JavaScript,
   extendedchars=true,
   basicstyle=\footnotesize\ttfamily,
   showstringspaces=false,
   showspaces=false,
   numbers=left,
   numberstyle=\footnotesize,
   numbersep=9pt,
   tabsize=2,
   breaklines=true,
   showtabs=false,
   captionpos=b
}

\NewDocumentCommand{\codeword}{v}{%
\texttt{\textcolor{blue}{#1}}%
}

% -----------------------------------------------------------------

\pagestyle{fancy}
\fancyhf{}
\lhead{\rightmark}
\rhead{\textbf{\thepage}}
\fancyfoot{}
\setlength{\headheight}{12.5pt}

% Rimuove il numero di pagina all'inizio dei capitoli
\fancypagestyle{plain}{
  \fancyfoot{}
  \fancyhead{}
  \renewcommand{\headrulewidth}{0pt}
}



% Modifica dello stile dei riferimenti, con il testo in cyano
\hypersetup{
    colorlinks,
    linkcolor=CornflowerBlue,
    citecolor=CornflowerBlue
}

\newtheorem{definition}{Definizione}[section]
\newtheorem{theorem}{Teorema}[section]
\providecommand*\definitionautorefname{Definizione}
\providecommand*\theoremautorefname{Teorema}
\providecommand*{\listingautorefname}{Listing}
\providecommand*\lstnumberautorefname{Linea}

\raggedbottom




% -----------------------------------------------------------------
\begin{document}

\begin{titlepage}
\begin{figure}[!htb]
    \centering
    \includegraphics[keepaspectratio=true,scale=0.5]{images/cherubino.eps}
\end{figure}

\begin{center}
    \LARGE{UNIVERSITÀ DI PISA}
    \vspace{5mm}
    \\ \large{DIPARTIMENTO DI INGEGNERIA DELL'INFORMAZIONE}
    \vspace{5mm}
    \\ \LARGE{Laurea Triennale in Ingegneria Informatica}
\end{center}

\vspace{15mm}
\begin{center}
    {\LARGE{\bf Progetto e realizzazione di una estensione VSCode per il debugging di un nucleo multiprogrammato}
    
\end{center}

\vspace{30mm}

\begin{minipage}[t]{0.47\textwidth}
	{\large{Relatore:}{\normalsize\vspace{3mm}
	\bf\\ \large{Prof: Giuseppe Lettieri} \normalsize\vspace{3mm}\bf \\ \large{Prof: Luigi Leonardi}}}
\end{minipage}
\hfill
\begin{minipage}[t]{0.47\textwidth}\raggedleft
	{\large{Candidato:}{\normalsize\vspace{3mm} \bf\\ \large{Francesco Mignone}}}
\end{minipage}

\vspace{30mm}
\hrulefill
\\\centering{\large{ANNO ACCADEMICO 2023/2024}}

\end{titlepage}


\begin{center}
    \LARGE{\bf Abstract}
    \vspace{5mm}
\end{center}

Questo elaborato descrive la progettazione e l'implementazione di un'estensione per VS Code che facilita il debug
del nucleo multiprogrammato didattico. Gli obiettivi principali dell'estensione includono la possibilità di
impostare e gestire breakpoint,
visualizzare variabili in tempo reale e eseguire codice passo-passo. Per raggiungere questi
obiettivi, l'estensione utilizza il Debug Adapter Protocol (DAP) per interfacciarsi con gli strumenti di debug
tradizionali come GDB.

Il lavoro presentato in questa tesi comprende un'analisi dettagliata dei requisiti funzionali e non funzionali,
l'analisi dell'architettura dell'estensione e l'implementazione delle principali funzionalità.
Lo scopo dell'estensione è semplificare significativamente il processo di debug del nucleo
offrendo un'interfaccia utente intuitiva e funzionalità avanzate di debug.

Vengono inoltre discusse le limitazioni dell'estensione e vengono proposte possibili direzioni per futuri miglioramenti,
tra cui l'aggiunta di ulteriori funzionalità e l'ottimizzazione delle prestazioni.

\tableofcontents
\setcounter{page}{1}
\chapter{Introduzione}
Il nucleo di un sistema operativo è il componente software fondamentale che gestisce le risorse hardware e software. A causa della sua complessità e della sua importanza critica, il debug del nucleo è un compito altamente specialistico che richiede strumenti potenti e una profonda comprensione del sistema. Tuttavia, gli strumenti tradizionali come GDB e QEMU possono essere difficili da utilizzare, soprattutto per i nuovi sviluppatori o per coloro che preferiscono interfacce grafiche più intuitive.

Visual Studio Code (VSCode) è un IDE open-source sviluppato da Microsoft. VS Code è diventato molto popolare tra gli sviluppatori grazie alla sua facilità d'uso e alla sua capacità di supportare molti linguaggi di programmazione grazie alla sua vasta gamma di estensioni disponibili. Tuttavia, al momento della scrittura di questa tesi, non esiste un'estensione dedicata per il debug del nucleo didattico su VSCode.

L'obiettivo di questa tesi è sviluppare un'estensione per VS Code che renda il debug del nucleo più accessibile e intuitivo, permettendo agli sviluppatori di beneficiare dell'ambiente user-friendly di VSCode senza rinunciare alla potenza degli strumenti tradizionali.

\chapter{Ambiente e strumenti}
\section{Il debugger - GDB e QEMU}
\subsection{Quick emulator - QEMU}
QEMU è un emulatore open-source che permette di emulare l'architettura di un calcolatore. L'emulatore è configurato per emulare un PC dotato di processore AMD64 su cui viene avviato il Nucleo Didattico.

\subsection{GNU Debugger - GDB}
GNU Debugger (GDB) è un debugger portatile, permette quindi di testare e effettuare il debug di programmi. Eseguire il programma in questo ambiente controllato permette al programmatore di tenere traccia dell'esecuzione e monitorare le risorse al fine di individuare un eventuale malfunzionamento nel codice. Per la realizzazione dell'estensione utilizzeremo la funzionalità di debug remoto per connetterci ad un socket di sistema utilizzato da QEMU per il debug. GDB utilizza delle chiamate di sistema chiamate \codeword{process trace} (\codeword{ptrace}). 

\subsubsection{I breakpoint}
Un breakpoint permette al programma in esecuzione all'interno di un debugger di interrompere il flusso in un determinato punto. Si realizza sostituendo all'istruzione alla quale si vuole fermare l'esecuzione, una speciale istruzione la quale solitamente invia un segnale SIGTRAP, il quale verrà catturato dal debugger. Il procedimento di sostituzione è eseguito dal debugger stesso prima di avviare l'esecuzione, nel caso di GDB il programmatore deve eseguire il comando \codeword{break [arg]} dove l'argomento può essere la specifica linea di codice o un simbolo. 

\subsubsection{Continue e Stop}
Tramite i comandi \codeword{continue} e \codeword{stop} possiamo rispettivamente, a seguito di un interruzione, continuare la normale esecuzione del codice oppure interrompere l'esecuzione del programma.

\subsubsection{Step Over}
Permette di proseguire alla prossima istruzione senza entrare nei componenti interni dell'istruzione a cui siamo fermi attualmente.

\subsubsection{Step Into}
Rende possibile seguire il codice riga-per-riga entrando anche nei componenti interni e subroutine.

\subsubsection{Step Out}
Quando il programma è arrestato all'interno di una subroutine e si vuole risalire al chiamante, il comando \codeword{stepOut} permette di far continuare l'esecuzione fino a ritornare all'istruzione successiva del chiamante.

\subsubsection{Analisi delle variabili}
Durante l'esecuzione vi può essere la necessità di osservare come il valore o il tipo di una variabile vari. Inoltre è possibile modificare il valore delle variabili sul momento ad esecuzione in corso.

\subsubsection{Call Stack}
Il call stack, o program stack, è una struttura che permette di raccogliere informazioni su tutte le subroutine di un programma in esecuzione. Tale struttura è utile per tenere traccia di quale routine ha il controllo del flusso di istruzioni e a chi deve restituire tale controllo al termine della propria esecuzione.

\subsubsection{Comandi personalizzati}
Ulteriore funzione di GDB è la possibilità di estendere le funzionalità, tramite script in python, come la creazione di comandi personalizzati.

\section{L'architettura del debugger di VS Code}
In generale, se si volesse creare uno strumento di debug grafico, è necessario implementare da zero l'intera parte grafica e
non sarebbe possibile integrarla con gli IDE già esistenti. Per risolvere questo problema gli sviluppatori di IDE hanno realizzato
un livello intermedio che permette la comunicazione tra l'IDE e gli strumenti di debug, in particolare VSCode ha realizzato un protocollo che permette di comunicare con i debugger: il Debug Adapter Protocol (DAP). VSCode tramite il DAP si interfaccia non direttamente al debugger ma a un attore intermedio, il Debug Adapter (DA) il quale si occupa di trasformare le richieste dell'applicazione in comandi per il debugger in uso.

\begin{figure}[H]
    \centering
    \includegraphics[width=0.7\columnwidth]{images/debug-arch1.png}
    \caption{DAP e DP}
    \label{fig:dap e dp}
\end{figure}

Rende possibile la realizzazione di una generica interfaccia di debug la quale poi si occupa di comunicare con uno o più DP. Inoltre i Debug Adapter possono essere riutilizzati in diversi ambienti di sviluppo, eliminando la necessità di crearne uno specifico per ogni esigenza. 

\begin{figure}[H]
    \centering
    \includegraphics[width=0.7\columnwidth]{images/with-DAP.png}
    \caption{Ambienti di sviluppo multipli}
    \label{fig:Ambienti di sviluppo multipli}
\end{figure}

\subsection{Debug Adapter Protocol e Debug Adapter}

Analizziamo come avviene la connessione e scambio di messaggi tra l'applicativo e il debugger. Gli strumenti di sviluppo possono interagire con il Debug Adapter in due modi:
\begin{itemize}
    \item {
        Modalità a singola sessione: l'applicazione avvia una sessione di debug singola e comunica attraverso \codeword{stdin} e \codeword{stdout}. Alla fine della sessione, il Debug Adapter viene terminato
    }
    \item {
        Modalità a sessioni multiple: l'applicazione di debug si connette a un debugger già avviato in precedenza e si disconnette al termine della sessione.
    }
\end{itemize}


Il DAP supporta molte funzionalità, ciascuna rappresentata da una "capacità". Quando inizia una sessione di debug, lo strumento di sviluppo invia una richiesta di inizializzazione per reperire le capacità dell'adattatore. Dopo l'inizializzazione, il Debug Adapter è pronto per accettare richieste di avvio o collegamento.
\subsubsection*{Breakpoint}
Lo strumento di sviluppo gestisce i breakpoint inviando le informazioni di configurazione all'adattatore prima dell'esecuzione del programma. Quando il programma si ferma, l'adattatore, solitamente, invia un evento di stop con il motivo e l'id del thread. Lo strumento di sviluppo richiede i thread e lo stacktrace, e tramite essi risale alle variabili.

\subsubsection*{Inizio della sessione di debug}
Dopo aver stabilito una connessione, lo strumento di sviluppo comunica con l'adattatore tramite il protocollo di base. Il protocollo di base è implementato tramite lo scambio di messaggi composti da un'intestazione e un contenuto, chiamati \codeword{ProtocolMessage}. 

\begin{figure}[H]    
    \centering
    \includegraphics[width=0.7\columnwidth]{images/init-launch.png}
    \caption{Esempio di avvio di una sessione di debug\cite{DAP}}
\end{figure}

Quando inizia una sessione di debug, lo strumento di sviluppo deve comunicare con il Debug Adapter che implementa il Debug Adapter Protocol. Il protocollo di base è implementato tramite lo scambio di messaggi composti. Questi messaggi sono composti da un'intestazione e un contenuto, chiamati \codeword{ProtocolMessage}. 
\begin{figure}[H]
    \lstinputlisting[language=JavaScript]{code/protocolMessage.txt}
    \caption{ProtocolMessage\cite{DAPmessage}}
\end{figure}

\subsubsection*{Termine della sessione di debug}

Il processo per terminare la sessione è diverso a seconda di come si è stata avviata la sessione, "avviato" o "agganciata":

\begin{itemize}
    \item {
        debugger "avviato": se il Debug Adapter implementa la richiesta di interruzione, allora la sessione viene terminata correttamente. Se non dovesse essere supportata, la sessione continua a essere attiva fino a quando il debugger stesso non invia il comando di terminazione forzata
    }
    \item {
        debugger "agganciato": l'applicazione di debug  invia una richiesta di disconnessione al Debug Adapter. Questo permette al debugger di cessare la connessione con l'applicativo e continuare l'esecuzione
    }
\end{itemize}

se la sessione di debug termina, e il Debug Adapter è opportunamente configurato, un messaggio di corretta terminazione di sessione viene inviato all'applicazione di debug del programmatore.

\begin{figure}[H]    
    \centering
    \includegraphics[width=0.7\columnwidth]{images/stop-continue-terminate.png}
    \caption{Esempio terminazione di una sessione di debug\cite{DAP}}
\end{figure}

\subsection{Logpoints}
I logpoint sono una variante dei breakpoint. Permettono, senza interrompere l'esecuzione, di controllare il valore di una o più variabili e mostrando il risultato nella console di debug di VSCode. Sono molto utili per evitare aggiungere codice di log all'interno del programma.

\section{Webview di VS Code}

VSCode mette a disposizione la possibilità di creare nuove schede nelle quali un utente può visualizzare contenuti personalizzati. Le webview sono molto simili a degli \codeword{iframe} e sono capaci di disegnare qualsiasi contenuto HTML. 

\begin{figure}[H]
    \centering
    \includegraphics[width=0.7\columnwidth]{images/cat_coding.png}
    \caption{Esempio di una webview in VSCode}
    \label{fig:webcat}
\end{figure}

\section{Requisiti dell'estensione}
Le funzionalità richieste per la prima versione dell'estensione sono l'implementazione delle operazioni di base di debug: continue, stop, step over, step into, step out e l'analisi delle variabili. Inoltre è richiesto mostrare una lista dei processi attualmente in esecuzione sul Nucleo.

\chapter{Implementazione}
\section{Event listener}
ci scrivi del onDebugSessionStart

\section{Webview}
ci scrivi della classe e i suoi metodi
\subsection{Update Webview}
\subsection{Custom Debug Adapter request}

\chapter{Utilizzo del debugger}
Introduciamo ora come é possibile utilizzare l'ambiente di debug di VSCode per eseguire il debug del nucleo. È possibile avviare l'interfaccia di debug tramite l'apposita scheda \ref{fig:startDebug} (pin 1) e poi dopo aver selezionato la configurazione \codeword{launch nmd}  dal menù a tendina \ref{fig:startDebug} (pin 2)si avvia la sessione premendo il tasto di avvio collocato accanto allo stesso menù.  

\begin{figure}[H]
    \centering
    \includegraphics[height=0.3\pdfpageheight]{images/startDebug.png}
    \caption{Azioni per avviare il debug}
    \label{fig:startDebug}
\end{figure}

oppure premendo il tasto \codeword{F5} sulla tastiera.

l'interfaccia che ci viene presentata é la seguente 

\begin{figure}[H]
    \centering
    \includegraphics[width=\columnwidth]{images/fullDebug.png}
    \caption{Interfaccia di debug di  VSCode}
    \label{fig:debug screen}
\end{figure}

Analizziamo le varie finestre che ci vengono proposte.

\section{Breakpoint e Logpoint}
All'interno di VSCode vi é la possibilità di inserire breakpoint cliccando sul lato sinistro della riga di codice interessata, VSCode stesso si occuperá poi di comunicare al Debug Adapter la richiesta di inserimento del breakpoint.

\begin{figure}[H]
    \centering
    \includegraphics[width=0.7\columnwidth]{images/breakpoint_vscode.png}
    \caption{Esempio di un breakpoint in VSCode}
    \label{fig:breakpoint}
\end{figure}

Similarmente si può inserire un logpoint cliccando con il tasto destro del mouse e selezionando la dicitura logpoint. All'interno del box di testo si possono aggiungere le variabili da osservare tramite \codeword{{{variable}}}

\begin{figure}[H]
    \centering
    \includegraphics[width=0.7\columnwidth]{images/logpoint.png}
    \caption{Esempio di logpoint}
    \label{fig:logpoint}
\end{figure}

una volta ripresa l'esecuzione del codice possiamo vedere l'output nella console di debug.

\section{Azioni di debug}

VSCode mette a disposizione una barra di funzioni \ref*{fig:middleDebug}(pin 1) per permettere all'utente di ispezionare il codice tramite i comandi di step over, step in e step out. È possibile anche eseguire azioni come continue, stop e il riavvio della sessione di debug.  

\begin{figure}[H]
    \centering
    \includegraphics[width=0.8\columnwidth]{images/middle_debug.png}
    \caption{Azioni e linea di comando}
    \label{fig:middleDebug}
\end{figure}

Inoltre é possibile inviare comandi di GDB direttamente dalla linea di comando \ref*{fig:middleDebug}(pin 2) tramite il comando \codeword{-exec [GDB command]}.

\section{Pannello di sinistra}

Il pannello di sinistra permette di analizzare lo stato delle variabili locali \ref*{fig:leftDebug}(pin 2) e lo stato dei registri \ref*{fig:leftDebug}(pin 3).

\begin{figure}[H]
    \centering
    \includegraphics[height=0.4\pdfpageheight]{images/leftDebug.png}
    \caption{Azioni e linea di comando}
    \label{fig:leftDebug}
\end{figure}

\subsection*{Watch}
Nel pannello dedicato \ref*{fig:leftDebug}(pin 4) possiamo visualizzare le variabili messe sotto osservazione. L'aggiunta puó avvenire tramite il pulsante "+" oppure selezionando con il cursore la variabile o l'espressione da analizzare e cliccando con il tasto destro dal menú a tendina selezionare "Add to watch" come mostrato in figura.

\begin{figure}[H]
    \centering
    \includegraphics[width=0.8\columnwidth]{images/addWatch.png}
    \caption{Aggiunta al watch di una variabile}
    \label{fig:watchvariabili}
\end{figure}

\subsection*{Call stack e breakpoints}
È inoltre possibile visualizzare informazioni riguardanti il call stack \ref*{fig:leftDebug}(pin 5) e gestire i breakpoint presenti tramite il pannello dedicato \ref*{fig:leftDebug}(pin 6)

\section{Informazioni aggiuntive}

Sulla parte destra della finestra di debug troviamo la scheda dedicata alle informazioni aggiuntive del nucleo. In questo caso
mostra solo i processi in secuzione 

\begin{figure}[H]
    \centering
    \includegraphics[height=0.3\pdfpageheight]{images/infoNucleo.png}
    \caption{Informazioni aggiuntive sul nucleo}
    \label{fig:infoNucleo}
\end{figure}

\subsection*{Processi in esecuzione}
Le informazioni relative ai processi vengono mostrate tramite una lista con elementi selezionabili. Ogni elemento della lista é formato da una rappresentazione tramite \codeword{chiave:valore} e un eventuale campo adiacente con eventuali informazioni sulla tipologia della variabile o oggetto. I processi sono stati divisi per convenienza tra processi di sistema e processi utente. 




\chapter{Conclusione e sviluppi futuri}
L'utilizzo estensione rende semplice approcciarsi al debug e in particolare al debug del nucleo didattico. Tuttavia l'estensione non è completa, vi sono numerose funzionalitá che possono essere implementate per ampliare le informazioni relative al nucleo come informazioni riguardanti la memoria virtuale o stato dell'APIC.

\chapter{Sviluppo e contributi}
In questo capitolo spiego come é possibile aggiungere nuove funzionalità all'estensione. È possibile sviluppare l'estensione tramite la macchina virtuale fornita durante il corso, tuttavia si consiglia vivamente di installare il nucleo sulla propria macchina seguendo le istruzioni fornite dal Professor Lettieri\cite{mainsite}, come editor é molto indicato VSCode stesso poiché é configurato per debug e esecuzione delle estensioni.

\section*{Installazione}
Per prima cosa bisogna impostare l'ambiente di sviluppo per l'estensione:
\begin{itemize}
    \item scaricare la repository del progetto \href{google.com}{Nucleo-Debugger} 
    \item navigare allinterno della repository
    \item creare la propria branch di sviluppo della feature da implementare tramite \codeword{git}
    \item installare le dipendenze: \codeword{yarn} e \codeword{node}
    \item inizializzare l'ambiente con il comando \codeword{yarn install} e una volta terminato lanciare il comando \codeword{yarn global add vsce} per installare il pacchetto di VSCode per creare il file di installazione dell'estensione
    \item aprire VSCode con il comando \codeword{code .}
\end{itemize}

Se eventualmente l'estensione per il debug del nucleo dovesse essere presente tra le estensioni di VSCode bisogna rimuoverla.

\subsection*{Debug e esecuzione}
È possibile caricare l'estensione in un enviroment separato da quello di sviluppo tramite proprio il debugger di VSCode premendo il tasto \codeword{F5}. Una volta completata la compilazione verrá avviata una nuova finestra di VSCode dove apriamo la cartella di un nucleo, in questo caso possiamo usare \codeword{prova-test} e avviare a sua volta il debug tramite \codeword{F5}.

\section{Aggiunta di comandi}
Per aggiungere comandi bisogna modificare il file \codeword{NucleoInfo.ts} nelle sezioni indicate:
\begin{itemize}
    \item {
        se vi é la necessitá di ricevere dati da GDB bisogna dichiarare una variabile di supporto.
        \begin{figure}[H]
            \lstinputlisting[language=JavaScript]{code/var.txt}
            \caption{Variabile di supporto per i dati}
        \end{figure}
    }
    \item {
        Per eseguire un comando é necessario chiamare la funzione \linebreak \codeword{.customCommand(args ...)} e se si devono ricevere dati assegnarli alla relativa variabiel di appoggio.
        \begin{figure}[H]
            \lstinputlisting[language=JavaScript]{code/command_exec.txt}
            \caption{Richiesta di esecuzione del comando}
        \end{figure}
    }
    \item {
        Per aggiornare la webview bisogna aggiungere all'HTML giá presente il proprio tramite  una funzione di appoggio che si occupa di formattare i dati (es. \codeword{this.YOUR_FORMATTER()})
        \begin{figure}[H]
            \lstinputlisting[language=JavaScript]{code/getHTML.txt}
            \caption{Costruzione della pagina HTML}
        \end{figure}
    }
    \item {
        il formattatore deve creare il codice HTML da incormporare nella funzione \codeword{_getHtmlForWebview()} per mostrare correttamente i dati
        \begin{figure}[H]
            \lstinputlisting[language=JavaScript]{code/formatter.txt}
            \caption{Implementazione di un formattatore}
        \end{figure}
        come spiegato in seguito, le strutture per lo scambio di informazioni sono a discrezione del programmatore, tuttavia si consiglia l'utilizzo del formato \codeword{JSON} per faciliare la visualizzazione tramite il tool di templating handlebars\cite{handlebars}.
    }
\end{itemize}
\subsection{nucleo\textunderscore vscode.py}
All'interno del file \codeword{nucleo_vscode.py} bisogna implementare il nuovo comando richiesto dall'estensione. Per l'effettiva implementazione del comando dipende dalla funzione che si vuole aggiungere, si faccia riferimento alla guida per lo scripting di GDB\cite{GDBpython} e al file \codeword{./debug/nucleo.py} all'interno di una delle versioni del nucleo presenti sul sito del Professor Lettieri \cite{testiEsame}. 

% \href[options]{https://github.com/microsoft/vscode-cpptools/blob/main/Documentation/Building%20the%20Extension.md}{Build istructions microsoft}

%\href[options]{https://stackoverflow.com/questions/56237448/how-to-make-acquirevscodeapi-available-in-vscode-webview-react}{se si vuole le API di vscode nel javascript}
%\href[options]{https://www.codemag.com/Article/1809051/Writing-Your-Own-Debugger-and-Language-Extensions-with-Visual-Studio-Code}{Come scrivere il proprio language support e debug}
%\href[options]{https://stackoverflow.com/questions/59040679/how-to-output-stopped-events-in-visual-studio-code-debugger#59068740}{implementare lo stopped event}
%
%\href[options]{https://stackoverflow.com/questions/56012353/how-to-get-vs-code-debug-data-like-breakpoints-steps-line-code/63657824#63657824}{Debug tracker}
%\href[options]{https://github.com/Microsoft/vscode/issues/62843}{Debug tracker API}
%
%\href[options]{https://stackoverflow.com/questions/59040679/how-to-output-stopped-events-in-visual-studio-code-debugger}{stopOnStep listener}



% Rimuovere se non si vuole la tabella delle figure
% \listoffigures

\include{chapters/ringrazia}

\appendix
\bibliography{chapters/bibliografia.bib}
\bibliographystyle{plainnat}

\end{document}
% -----------------------------------------------------------------
