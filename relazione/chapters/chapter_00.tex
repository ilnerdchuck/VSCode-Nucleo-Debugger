\begin{center}
    \LARGE{\bf Abstract}
    \vspace{5mm}
\end{center}

Questo elaborato descrive la progettazione e l'implementazione di un'estensione per VS Code che facilita il debug
del nucleo multiprogrammato didattico. Gli obiettivi principali dell'estensione includono la possibilità di
impostare e gestire breakpoint,
visualizzare variabili in tempo reale e eseguire codice passo-passo. Per raggiungere questi
obiettivi, l'estensione utilizza il Debug Adapter Protocol (DAP) per interfacciarsi con gli strumenti di debug
tradizionali come GDB.

Il lavoro presentato in questa tesi comprende un'analisi dettagliata dei requisiti funzionali e non funzionali,
l'analisi dell'architettura dell'estensione e l'implementazione delle principali funzionalità.
Lo scopo dell'estensione è semplificare significativamente il processo di debug del nucleo
offrendo un'interfaccia utente intuitiva e funzionalità avanzate di debug.

Vengono inoltre discusse le limitazioni dell'estensione e vengono proposte possibili direzioni per futuri miglioramenti,
tra cui l'aggiunta di ulteriori funzionalità e l'ottimizzazione delle prestazioni.
