\chapter{Conclusione e sviluppi futuri}
L'utilizzo estensione rende semplice approcciarsi al debug e in particolare al debug del nucleo didattico. Tuttavia l'estensione non è completa, vi sono numerose funzionalitá che possono essere implementate per ampliare le informazioni relative al nucleo come informazioni riguardanti la memoria virtuale o stato dell'APIC.

\chapter{Sviluppo e contributi}
In questo capitolo spiego come é possibile aggiungere nuove funzionalità all'estensione. È possibile sviluppare l'estensione tramite la macchina virtuale fornita durante il corso, tuttavia si consiglia vivamente di installare il nucleo sulla propria macchina seguendo le istruzioni fornite dal Professor Lettieri\cite{mainsite}, come editor é molto indicato VSCode stesso poiché é configurato per debug e esecuzione delle estensioni.

\section*{Installazione}
Per prima cosa bisogna impostare l'ambiente di sviluppo per l'estensione:
\begin{itemize}
    \item scaricare la repository del progetto \href{google.com}{Nucleo-Debugger} 
    \item navigare allinterno della repository
    \item creare la propria branch di sviluppo della feature da implementare tramite \codeword{git}
    \item installare le dipendenze: \codeword{yarn} e \codeword{node}
    \item inizializzare l'ambiente con il comando \codeword{yarn install} e una volta terminato lanciare il comando \codeword{yarn global add vsce} per installare il pacchetto di VSCode per creare il file di installazione dell'estensione
    \item aprire VSCode con il comando \codeword{code .}
\end{itemize}

Se eventualmente l'estensione per il debug del nucleo dovesse essere presente tra le estensioni di VSCode bisogna rimuoverla.

\subsection*{Debug e esecuzione}
È possibile caricare l'estensione in un enviroment separato da quello di sviluppo tramite proprio il debugger di VSCode premendo il tasto \codeword{F5}. Una volta completata la compilazione verrá avviata una nuova finestra di VSCode dove apriamo la cartella di un nucleo, in questo caso possiamo usare \codeword{prova-test} e avviare a sua volta il debug tramite \codeword{F5}.

\section{Aggiunta di comandi}
Per aggiungere comandi bisogna modificare il file \codeword{NucleoInfo.ts} nelle sezioni indicate:
\begin{itemize}
    \item {
        se vi é la necessitá di ricevere dati da GDB bisogna dichiarare una variabile di supporto.
        \begin{figure}[H]
            \lstinputlisting[language=JavaScript]{code/var.txt}
            \caption{Variabile di supporto per i dati}
        \end{figure}
    }
    \item {
        Per eseguire un comando é necessario chiamare la funzione \linebreak \codeword{.customCommand(args ...)} e se si devono ricevere dati assegnarli alla relativa variabiel di appoggio.
        \begin{figure}[H]
            \lstinputlisting[language=JavaScript]{code/command_exec.txt}
            \caption{Richiesta di esecuzione del comando}
        \end{figure}
    }
    \item {
        Per aggiornare la webview bisogna aggiungere all'HTML giá presente il proprio tramite  una funzione di appoggio che si occupa di formattare i dati (es. \codeword{this.YOUR_FORMATTER()})
        \begin{figure}[H]
            \lstinputlisting[language=JavaScript]{code/getHTML.txt}
            \caption{Costruzione della pagina HTML}
        \end{figure}
    }
    \item {
        il formattatore deve creare il codice HTML da incormporare nella funzione \codeword{_getHtmlForWebview()} per mostrare correttamente i dati
        \begin{figure}[H]
            \lstinputlisting[language=JavaScript]{code/formatter.txt}
            \caption{Implementazione di un formattatore}
        \end{figure}
        come spiegato in seguito, le strutture per lo scambio di informazioni sono a discrezione del programmatore, tuttavia si consiglia l'utilizzo del formato \codeword{JSON} per faciliare la visualizzazione tramite il tool di templating handlebars\cite{handlebars}.
    }
\end{itemize}
\subsection{nucleo\textunderscore vscode.py}
All'interno del file \codeword{nucleo_vscode.py} bisogna implementare il nuovo comando richiesto dall'estensione. Per l'effettiva implementazione del comando dipende dalla funzione che si vuole aggiungere, si faccia riferimento alla guida per lo scripting di GDB\cite{GDBpython} e al file \codeword{./debug/nucleo.py} all'interno di una delle versioni del nucleo presenti sul sito del Professor Lettieri \cite{testiEsame}. 

% \href[options]{https://github.com/microsoft/vscode-cpptools/blob/main/Documentation/Building%20the%20Extension.md}{Build istructions microsoft}

%\href[options]{https://stackoverflow.com/questions/56237448/how-to-make-acquirevscodeapi-available-in-vscode-webview-react}{se si vuole le API di vscode nel javascript}
%\href[options]{https://www.codemag.com/Article/1809051/Writing-Your-Own-Debugger-and-Language-Extensions-with-Visual-Studio-Code}{Come scrivere il proprio language support e debug}
%\href[options]{https://stackoverflow.com/questions/59040679/how-to-output-stopped-events-in-visual-studio-code-debugger#59068740}{implementare lo stopped event}
%
%\href[options]{https://stackoverflow.com/questions/56012353/how-to-get-vs-code-debug-data-like-breakpoints-steps-line-code/63657824#63657824}{Debug tracker}
%\href[options]{https://github.com/Microsoft/vscode/issues/62843}{Debug tracker API}
%
%\href[options]{https://stackoverflow.com/questions/59040679/how-to-output-stopped-events-in-visual-studio-code-debugger}{stopOnStep listener}
