Aggiungiamo al nucleo il meccanismo delle {\em barriere con timeout}.

Una barriera {\em senza} timeout serve a sincronizzare un certo
numero di processi e
funziona nel modo seguente: la barriera \`e normalmente chiusa;
un processo che arriva alla barriera si blocca;
la barriera si apre solo quando sono arrivati tutti i processi attesi, che a
quel punto si sbloccano; una volta aperta e sbloccati tutti i processi,
la barriera si richiude e il meccanismo si ripete.

Il timeout cambia le cose nel seguente modo:
\begin{itemize}
  \item il primo processo che arriva alla barriera dopo una chiusura fa partire il timeout;
  \item se tutti gli altri processi attesi arrivano prima dello scatto
    del timeout, la barriera si comporta normalmente (si apre, tutti
    i processi si sbloccano e poi la barriera si richiude);
  \item altrimenti la barriera entra in uno stato ``erroneo'':
    si apre (quindi,  processi gi\`a arrivati 
    si risvegliano) e resta aperta fino a quando non sono arrivati
    tutti i processi attesi, ma tutti i processi la attraversano
    ricevendo un errore; quando arriva l'ultimo processo, la barriera
    esce dallo stato erroneo e si richiude.
\end{itemize}
{\bf Nota}: sopra e nel seguito, dove diciamo ``dall ultima chiusura'', intendiamo
anche l'istante in cui la barriera \`e stata creata.

Per rappresentare una barriera introduciamo la seguente struttura dati:
\begin{verbatim}
    struct barrier_t {
        natl nproc;
        natl narrived;
        natl timeout;
        bool bad;
        des_proc *waiting;
        des_proc *first;
    };
\end{verbatim}
Dove: \verb|nproc| \`e il numero di processi che devono sincronizzarsi sulla barriera;
\verb|narrived| conta i processi arrivati alla barriera dall'ultima chiusura;
\verb|timeout| \`e il timeout che regola l'entrata
nello stato erroneo; \verb|bad| \`e true se e solo se la barriera si trova nello stato erroneo;
\verb|waiting| \`e la coda dei processi che attendono
l'apertura della barriera; \verb|first| punta al descrittore del primo processo arrivato dall'ultima chiusura
(quello che ha avviato il timeout corrente).

Aggiungiamo anche il seguente campo ai descrittori di processo:
\begin{verbatim}
    natl barrier_id;
\end{verbatim}
Se questo campo \`e diverso da 0xFFFFFFFF, vuol dire che questo processo \`e bloccato
sulla barriera con identificatore \verb|barrier_id|, ed \`e il primo processo ad essere arrivato
su quella barriera dall'ultima chiusura.

Aggiungiamo inoltre le seguenti primitive:
\begin{itemize}
  \item \verb|natl barrier_create(natl nproc, natl timeout)| (gi\`a realizzata):
   	crea una nuova barriera che sincronizza \verb|nproc| processi con timeout \verb|timeout|
	e ne restituisce l'identificatore (\verb|0xFFFFFFFF| se non \`e stato possibile completare l'operazione).
   \item \verb|bool barrier(natl id)| (da realizzare):
   	fa giungere il processo corrente alla barriera di identificatore \verb|id|. \`E un errore
	se tale barriera non esiste. Restituisce \verb|true| quando termina normalmente, e \verb|false|
	quando termina perch\`e la barriera \`e stata attraversata nello stato erroneo.
\end{itemize}

Le primitive abortiscono il processo chiamante in caso di errore e tengono conto della priorit\`a tra i processi.

Modificare i file \verb|sistema.cpp| e \verb|sistema.s| in modo da realizzare le primitive mancanti.
